% Todas as linhas precedidas pelo simbolo '%' são comentários
% e não afetam em nada o seu texto final.

% IGNORE. Pacotes necessários e acessórios para o documento
\documentclass[12pt]{exam}
\usepackage{amsthm}
%\usepackage{libertine}
\usepackage[T1]{fontenc}
\usepackage[utf8]{inputenc}
\usepackage[margin=1in]{geometry}
\usepackage{amsmath,amssymb}
\usepackage{multicol}
\usepackage[shortlabels]{enumitem}
\usepackage{subcaption}
% ---
\usepackage{listings}
\usepackage{xcolor}
\usepackage{verbatim}
\usepackage{seqsplit}
\usepackage{titlesec}
\usepackage{changepage}
\usepackage{graphicx}
\usepackage{float}

\graphicspath{ {./images/} }



\newcommand{\class}{CPE 321 - Introduction to Security} 
\newcommand{\term}{Fall 2020}  
\newcommand{\worknum}{Assignment 2} 
\newcommand{\examdate}{20/21/0020} 


\setlength{\parindent}{0in}

\definecolor{codegreen}{rgb}{0,0.6,0}
\definecolor{codegray}{rgb}{0.5,0.5,0.5}
\definecolor{codepurple}{rgb}{0.58,0,0.82}
\definecolor{backcolour}{rgb}{0.95,0.95,0.92}
\definecolor{backcolour}{rgb}{0.97,0.97,0.95}

\lstdefinestyle{mystyle}{
    backgroundcolor=\color{backcolour},   
    commentstyle=\color{codegreen},
    keywordstyle=\color{magenta},
    numberstyle=\tiny\color{codegray},
    stringstyle=\color{codepurple},
    basicstyle=\ttfamily\footnotesize,
    breakatwhitespace=false,         
    breaklines=true,                 
    captionpos=b,                    
    keepspaces=true,                 
    %numbers=left,                    
    numbersep=5pt,                  
    showspaces=false,                
    showstringspaces=false,
    showtabs=false,                  
    tabsize=2,
    frame=single,
    framerule=1pt,
    xleftmargin=25pt,
    xrightmargin=25pt,
    breakindent=0pt,
    resetmargins=true,
    aboveskip=0pt, 
    framesep=0pt, 
    % Margins and box
    framextopmargin=15pt,
    framexbottommargin=15pt,
    framexleftmargin=15pt,
    framexrightmargin=15pt
}

\lstset{style=mystyle}

\titleformat{\section}
{\Large\bfseries}
{\thesection}{0.5em}{}

\titleformat{\subsection}
{\normalfont\ttfamily\bfseries}
{\large{\thesubsection) }}{0.5em}{}

\renewcommand{\thesection}{}% Remove section references...
\renewcommand{\thesubsection}{\arabic{subsection}}%... from subsections

\begin{document} 
\pagestyle{plain}
\thispagestyle{empty}

\noindent
\begin{tabular*}{\textwidth}{l @{\extracolsep{\fill}} r @{\extracolsep{6pt}} l}
 \textbf{\worknum} & \textbf{Name:} & \textit{Ethan Ahlquist}\\
\textbf{\class} &&\\
\end{tabular*}\\

\rule[2ex]{\textwidth}{2pt}

% ---

\bigskip
\textbf{\huge{Assignment 2: OTP}}

    
\section*{Command Usage:}

\bigskip

\texttt{otp.py [TASK]... [FILES]... }

\bigskip

Task Flags:
\begin{itemize}[label={}]
    \item \texttt{-1 ...}
    \item \texttt{-2 INFILE OUTFILE} 
    \item \texttt{-3 IN.bmp OUT.bmp}
    \item \texttt{-4 IN1.bmp IN2.bmp OUT.bmp}
\end{itemize}

\section*{Repeating Logic:}

\begin{enumerate}[\bf 1)]
    \item \textbf{\texttt{xor\_otp()}}

    All tasks used this function.

    Checked to see if the encryption key and text were of the same length.
    Then xor-ing the bytes between them.

    \bigskip
    \lstinputlisting[language=Python, firstline=20, lastline=27]{./tools.py}
    \bigskip

    \item \textbf{\texttt{xor\_bytes()}}

    This xor's the contenst of two bytes objects.

    \bigskip
    \lstinputlisting[language=Python, firstline=16, lastline=17]{./tools.py}
    \bigskip

    \item \textbf{\texttt{random\_bytes()}}

    This reads random bytes from \texttt{/dev/urandom} for a given size.

    This is used in most tasks to produce a random key.

    \bigskip
    \lstinputlisting[language=Python, firstline=12, lastline=13]{./tools.py}
    \bigskip

    \item \textbf{\texttt{checkDecryption()}}

    This is used in some tasks to validate to the user that two byte\_strings
    actually have the same value.

    \bigskip
    \lstinputlisting[language=Python, firstline=5, lastline=9]{./tools.py}
    \bigskip

\end{enumerate}


\section*{Tasks:}

\subsection{./otp.py -1}

    \bigskip
This task only tests to see if the xor functionality is 
working between byte strings. 

My only difficulty was removing TypeErrors.

    \bigskip
    \bigskip
    \lstinputlisting[language=Python, firstline=7, lastline=10]{./otp.py}
    \bigskip

    The output was just the hex dump of the xor contents. 
    \bigskip
    
Printing: 
\begin{verbatim}
    250f164c0a1b54441601015259071449154e
\end{verbatim}
    

\subsection{./otp.py -2 ./files/in.txt out.txt}

    \bigskip

This task was used to encrypt entire files, overwriting header information as
well as its contents. This required the file to be opened as a byte reader,
which took a little time to figure out.

    \bigskip
    \bigskip
    \lstinputlisting[language=Python, firstline=13, lastline=27]{./otp.py}
    \bigskip

The output for this task is hard to display, because most of the output
characters are unprintable. However within the program there is a function that
checks to see if the decryption was done correctly by decrypting the encrypted
text and comparing it to the original text.

This function would print:

\begin{verbatim}
    Valid Decryption:  True
\end{verbatim}

When decryption was reversible.



\subsection{./otp.py -3 ./files/mustang.bmp task3out.bmp}

\bigskip

This task added the difficulty of dealing with a byte-offset where the certain header
information, for a bmp file, would not be encrypted. Otherwise, the encryption was the same.

    \bigskip
    \bigskip
    \lstinputlisting[language=Python, firstline=30, lastline=52]{./otp.py}
    \bigskip

    Here is the output file created from this encryption:

    \begin{figure}[H]
        \centering
        \includegraphics[width=0.50\textwidth]{task3out}
        \caption{Resulting encryption bmp file}
        \label{fig:task3}
    \end{figure}

    \textit{View the images. What do you observe? Are you able to derive any
    useful information about from either of the encrypted images? What are the
    causes for what you observe?}

    From this image, I could see absolutely no pattern to its output. The colors
    have very little association to each other and have wide distribution. What
    caused this randomness was purely caused by the randomness of the key,
    which was read from \texttt{/dev/urandom}. This key being completely
    random, overshadowed the file contents which did have a pattern.

    \subsection{./otp.py -4 ./files/mustang.bmp ./files/cp-logo.bmp task4out.bmp}

\bigskip

This task required two files to be encrypted with the same key. After this, the
two files would be xor'ed to each other, which produced a file that was hardly
encrypted, and the contents of both files were easily determined. The purpose
of this task was to display the importance of not re using keys for encryption
since the files then have an easy was to be decrypted.

    \bigskip
    \bigskip
    \lstinputlisting[language=Python, firstline=55, lastline=79]{./otp.py}
    \bigskip

    Here is the output file created xor'ing the two file encryptions:

    \begin{figure}[H]
        \centering
        \includegraphics[width=0.50\textwidth]{task4out}
        \caption{bmp file of xor between encrypted files w/
        shared key}
        \label{fig:task4}
    \end{figure}

    \textit{View the output. Are you able to now derive any useful information about
    the original plaintexts from the resulting image? What are the causes for
what you observe?}\\

    Now looking at the output, we can see extremely useful information regarding
    the file contents. In fact, we can determine the contents of both files
    simply by looking at it, that being a mustang and the cal poly logo. This is
    because an xor operation is a reversible action if you know certain enough
    information. In this case we know that the same key was used twice on both
    files, meaning an xor between the two files would provide an unencrypted
    file with the original files xor'ed together.

    \section*{Conclusion:}

The purpose of this lab was to display how important the use of unique keys are
to an encryption  possess, especially an otp encryption. This is displayed in
how easy the breaking of the encryption was, simply by xor'ing files with
non-unique keys. From this, I can see that modern security solutions to encryption may
be far more complicated than a simple xor to a string.



\end{document}

